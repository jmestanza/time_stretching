\documentclass[main.tex]{subfiles}

\begin{document}


\begin{abstract}
Se ecualizó una señal en un enlace digital basándose en el 
esquema de filtrado de inversión de sistemas. Se implementaron 
NLMS y RLS, justificando la elección de los parámetros, consiguiendo
minimizar el bit error rate, la tasa de errores.
esquema de filtrado de inversión de sistemas. Se implementó
NLMS, justificando la elección de los parámetros, consiguiendo
minimizar el bit error rate, la tasa de errores. Luego se buscó de manera exitosa 
implementar RLS, consiguiendo aún mejores resultados para el problema.

\end{abstract}


\section{Introducción}
Se buscó ecualizar una señal en un enlace digital de comunicaciones. 
Los datos consistían en una secuencia de datos pseudoaleatoria 
codificada por Manchester, muestreada a una frecuencia de sampleo
 de 4 kHz a razón de 250 bps. Ante estas características, cada bit consiste de 
 16 muestras según la codificación mencionada. El canal, en este caso, conocido, 
 modificaba la señal dependiendo del posicionamiento
 aleatorio de dos pares de polos conjugados. En lo que respecta a estacionaridad, 
 el canal carece de la misma debido a sus características cambiantes en el tiempo.
La principal dificultad consistía entonces en la variabilidad del canal. En las 
siguientes figuras se puede notar el efecto de lo mencionado, siendo la 
señal en azul los bits enviados y en naranja lo recibido. \newline

\begin{figure}[H]
    \centering
    \includegraphics[scale=0.5]{imagenes/1.jpeg}
\end{figure}
\begin{figure}[H]
    \centering
    \includegraphics[scale=0.5]{imagenes/2.jpeg}
\end{figure}
\begin{figure}[H]
    \centering
    \includegraphics[scale=0.5]{imagenes/3.jpeg}\\
    \includegraphics[scale=0.5]{imagenes/4.jpeg}
    \caption{Señales obtenidas en el receptor, al pasar por el canal.}
\end{figure}

El proyecto consiste en aplicar un algoritmo de filtrado adaptativo
 para recuperar la señal transmitida asumiendo que no se conoce la entrada, 
 lo que simularía un enlace digital de comunicaciones.
 Un esquema de lo mencionado
se puede ver en la siguiente figura.

\begin{figure}[H]
    \centering
    \includegraphics[scale=0.5]{imagenes/preblock.png}
    \caption{Esquema del enlace digital.}
\end{figure}

Como criterio de validación del algoritmo como así de sus parámetros, se calculó el bit error rate (BER), la
cantidad de bits errados por unidad de tiempo.



\end{document}